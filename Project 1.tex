\documentclass[conference]{IEEEtran}

\usepackage{graphicx}   
\usepackage{amsmath}    
\usepackage{booktabs}   
\usepackage{url}      

\title{Predicting Superconducting Critical Temperature Using Regression Analysis}

\author{
  \IEEEauthorblockN{Miguel Ramirez}
  \IEEEauthorblockA{
    Department of Statistics and Data Science \\
    University of Central Florida \\
    Orlando, United States \\
    miramirez@knights.ucf.edu}
}

\begin{document}

\maketitle

\begin{abstract}
This project develops a regression model to predict the superconducting critical temperature 
based on features extracted from the chemical formula. We evaluate linear regression models, 
apply stepwise feature selection, and compare predictive performance.
\end{abstract}

\begin{IEEEkeywords}
Superconductor, regression, machine learning, UCI dataset
\end{IEEEkeywords}

\section{Introduction}
A superconductor is a material that allows electricity to flow without resistance. 
The temperature at which this occurs is called the critical temperature ($T_c$).
Predicting $T_c$ from a compound’s chemical composition is a long-standing challenge in materials science.

\section{Data}
We used the Superconductivity Data Set from the UCI Machine Learning Repository 
(\url{https://archive.ics.uci.edu/dataset/464/superconductivty+data}), 
which contains 21,263 superconductors, 81 predictor variables, and one target variable ($T_c$).

\section{Methodology}
We fit a linear regression model:
\begin{equation}
Y = \beta_0 + \beta_1X_1 + \cdots + \beta_pX_p + \epsilon
\end{equation}
where $Y$ is the transformed $T_c$ and $X_i$ are the predictor variables.

\subsection{Model Selection}
We applied stepwise regression using BIC to reduce the number of predictors.

\section{Results}
Table~\ref{tab:results} shows performance metrics for the final regression model.

\begin{table}[htbp]
  \caption{Model Performance}
  \centering
  \begin{tabular}{lcc}
    \toprule
    Metric & Train & Test \\
    \midrule
    $R^2$       & 0.78 & 0.77 \\
    MSE         & 2.51 & 2.56 \\
    \bottomrule
  \end{tabular}
  \label{tab:results}
\end{table}

\section{Conclusion}
The regression model explained about 78\% of the variation in $T_c$. 
Key predictors included atomic mass, valence, and thermal conductivity.

\begin{thebibliography}{00}
\bibitem{hamidieh2018}
K. Hamidieh, ``A data-driven statistical model for predicting the critical temperature of a superconductor,'' 
\emph{Computational Materials Science}, vol. 154, pp. 346--354, 2018.
\end{thebibliography}

\end{document}
